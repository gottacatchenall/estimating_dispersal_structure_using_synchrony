\documentclass[]{article}
\usepackage{caption}
\usepackage{tocloft}
\usepackage{amssymb,amsmath}
\usepackage{ifxetex,ifluatex}
\usepackage{fixltx2e} % provides \textsubscript

\usepackage{longtable}
\usepackage{graphicx}
\usepackage{tikz}

\setlength{\parindent}{0pt}
\setlength{\parskip}{6pt plus 2pt minus 1pt}
\setlength{\emergencystretch}{3em}  % prevent overfull lines
\setcounter{secnumdepth}{0}

\usepackage[left=1in,right=1in]{geometry}
\usepackage{float}
\floatplacement{figure}{h}


\setcounter{secnumdepth}{3}
\addtocontents{toc}{\setcounter{tocdepth}{2}}

\usepackage{sectsty}
\usepackage[normalem]{ulem}
\sectionfont{\rmfamily\bfseries\large}
\subsectionfont{\rmfamily\centering\upshape\normalsize}
\subsubsectionfont{\centering\normalsize}

\fontsize 12
\usepackage{hyperref}
\usepackage{lineno}
\linenumbers
\usepackage[font={footnotesize,sf}]{caption}
\usepackage{siunitx}

\usepackage{setspace}
\linespread{1.5}

\hyphenpenalty 1000 % no hyphens!

\raggedbottom

\usepackage{titling}
\pretitle{\LARGE}
\preauthor{\large}
\predate{\large}
\postdate{}


\usepackage{fancyhdr}
\pagestyle{fancy}
\lhead{  \nouppercase  \leftmark}
\rhead{\thepage}
\cfoot{}

\usepackage{authblk}

\title{A software package for estimating the dispersal structure of a landscape using the synchrony of population dynamics \\}
\author[1,2]{M.D. Catchen} 
\author[1]{S.M. Flaxman}
\affil[1]{\small{Department of Ecology and Evolutionary Biology, University of Colorado at Boulder}}
\affil[2]{\small{Department of Biology, McGill University}}
\date{\today}


\usepackage[square, numbers]{natbib}
\bibliographystyle{unsrtnat}
\begin{document}
\maketitle
\begin{abstract}

dang ol absrtact here

\end{abstract}
\clearpage
\tableofcontents
\clearpage
\section{Introduction}

Understanding the dispersal structure of a landscape is important.
    There are several ways of measuring landscape connectivity.    
    Most measures of this are structural connectivity, meaning they are derived from the spatial properties of the landscape
    Functional connectivity is useful because it relates a landscape to a specific process
We propose a method for measuring functional connectivity of population dynamics. We do this using synchrony. 

The correlation structure of the pairwise comparisons of each time-series encodes information about how those time-series could be related. 

What causes synchrony? Some combination of correlation in the environment and dispersal. 
We represent the dispersal structure of a landscape using a spatial graph. Here edges represent what we call 
the \textit{dispersal potential}, which is a matrix of edge-weights $\Phi$, where $\Phi_{ij}$ is the probability that an individual born at location $i$ reproduces at location $j$. This is the object of inference in our model. 

\section{The Model}

The goal of the model is to infer the dispersal potential $\Phi$ given some time-series of abundances $A_i(t)$ across a set of spatial locations, $L$.

We have a set of priors $P(\theta)$ and want to infer $P(\theta | A, L)$. Writing down the likelihood function, $P(A | \theta, L)$ is mostly intractable for real landscapes. So, we use a simulation model to generate a pseudolikelihood function. 


\subsection{A Generative Model of Candidate Dispersal Potentials}

What are the objects of inference, $\theta$? We know that the primary goal of this 
is to infer $\Phi$, which is a matrix where every row is a probability distribution.
Sure, we could set our priors to be ten different Dirichlet distributions, but this requires
inferring a lot separate parameters, and it does not incorporate any of our knowledge about the landscape--i.e. more distant locations tend to have less likely dispersal than closer locations.


Take $\alpha$, $\epsilon$, and dispersal kernel, and generate $\Phi$.

Instead, estimate $\alpha$ and $\epsilon$, and draw from them to generate a
posterior distribution of dispersal potentials, $P(\Phi | A(t), L)$.


\subsection{A Generative Model of Metapopulation Dynamics}

We use a stochastic-logistic model locally, where diffusion occurs according to a reaction-diffusion model.



\subsection{Measuring Synchrony}

To measure the synchrony between two abundance trajectories, we use the crosscorrelation function $CC$, which has long been used for this purpose. 


\subsection{Approximate Bayesian Computation (ABC) Sampler}

We then implement an Approximate Bayesian Computation (ABC) Sampler to infer
the values of the parameters $\theta$ given a set of locations $L$ and observations of 
abundances, $A(t)$. 

The goal of the sampler is to estimate $P(\theta | A(t), L)$. We do this by measuring a sample $\pi$ from this distribution, which 

The sampler runs a loop, where $\forall s \in \{0,1,\dots,N_s\}$: 
\begin{enumerate}
    \item Draws a candidate value of $\theta \sim \text{Priors}$
    \item Construct a dispersal potential from $\theta$.
    \item Runs the generative stochastic-logistic model for $N_t$ timesteps
    \item Computes the synchrony matrix $S$ for the stochastic-logistic generated data
    \item Computes the distance $d$ between the stochastic-logistic synchrony matrix $S$ and the empirical synchrony matrix $\hat{S}$
    \item Accepts $\pi[s] = \theta$ and w.p proportional to $s$ if $d  < \rho$, otherwise does nothing.
\end{enumerate}

Caveats about convergence etc. 



\section{Testing the Model's Performance}


\subsection{An IBM to generate test data}

How do we know that our model makes accurate estimates about the dispersal structure of a landscape? One case would be to test it against experimental data where dispersal is highly controlled. However, given the limitation on data of this type, we use a simulation model where dispersal takes place on a smaller scale than our model of inference. This allows us to know some `ground truth' information about the process generating a set of abundance data $A(t)$.

We know that in real systems the factors that contribute to the stochasticity of population dynamics are varied \cite{melbourne_2008}.
Here, we implement an IBM Ricker model with different levels of stochasticity \cite{melbourne}. 

Dispersal occurs stochastically on the level of the individual---for each individual in the model whether that individual migrates , 

occurs according to a `true` dispersal potential $\Phi$. Each node's dispersal potential is drawn from a Dirichlet. 


\subsection{How many temporal observations does it need? }

\subsection{What if there is variance in parameters across populations?}

\subsection{What if there are multiple factors contributing to demographic stochasticity?}



\section{Conclusion}

Replace stochastic-logistic and run it on covid data

\clearpage
{
\footnotesize
\bibliography{references}
}



\end{document}
