\documentclass[]{article}
\usepackage{caption}
\usepackage{tocloft}
\usepackage{amssymb,amsmath}
\usepackage{ifxetex,ifluatex}
\usepackage{fixltx2e} % provides \textsubscript

\usepackage{longtable}
\usepackage{graphicx}
\usepackage{tikz}

\setlength{\parindent}{0pt}
\setlength{\parskip}{6pt plus 2pt minus 1pt}
\setlength{\emergencystretch}{3em}  % prevent overfull lines
\setcounter{secnumdepth}{0}

\usepackage[left=1in,right=1in]{geometry}
\usepackage{float}
\floatplacement{figure}{h}


\setcounter{secnumdepth}{3}
\addtocontents{toc}{\setcounter{tocdepth}{2}}

\usepackage{sectsty}
\usepackage[normalem]{ulem}
\sectionfont{\rmfamily\bfseries\large}
\subsectionfont{\rmfamily\centering\upshape\normalsize}
\subsubsectionfont{\centering\normalsize}

\fontsize 12
\usepackage{hyperref}
\usepackage{lineno}
\linenumbers
\usepackage[font={footnotesize,sf}]{caption}
\usepackage{siunitx}

\usepackage{setspace}
\linespread{1.5}

\hyphenpenalty 1000 % no hyphens!

\raggedbottom

\usepackage{titling}
\pretitle{\LARGE}
\preauthor{\large}
\predate{\large}
\postdate{}


\usepackage{fancyhdr}
\pagestyle{fancy}
\lhead{  \nouppercase  \leftmark}
\rhead{\thepage}
\cfoot{}

\usepackage{authblk}

\title{A software package for estimating the dispersal structure of a landscape using the synchrony of population dynamics \\}
\author[1,2]{M.D. Catchen} 
\author[1]{S.M. Flaxman}
\affil[1]{\small{Department of Ecology and Evolutionary Biology, University of Colorado at Boulder}}
\affil[2]{\small{Department of Biology, McGill University}}
\date{\today}


\usepackage[square, numbers]{natbib}
\bibliographystyle{unsrtnat}
\begin{document}
\maketitle
\begin{abstract}

dang ol absrtact here

\end{abstract}
\clearpage
\tableofcontents
\clearpage
\section{Introduction}

\begin{itemize}
    \item Understanding the dispersal structure of a landscape is important
    \item Most measures of this are structural connectivity, meaning they are derived from the spatial properties of the landscape
    \item Functional connectivity is useful because it relates a landscape to a specific process
    \item We propose a method for measuring functional connectivity of population dynamics
\end{itemize}

\section{model}

\begin{itemize}
    \item abc sampler to infer dispersal potential
    \item uses isolation-by-distance to propose $\Phi$
    \item we use a stochastic logistic model to generate trajectories of abundances
    \item generate matrix of pairwise synchronies, $CC$
    \item measure distance of matrix from data 
\end{itemize}


\section{testing the models performance}

\begin{itemize}
    \item use an individual based model which incorporates stochasticity and a level below our SDE
    \item how much data does it take to get accurate estimates of the dispersal structure? 
\end{itemize}



\section{caveats}
\subsection{Caveats with things we are not inferring}
\begin{itemize}
    \item How does our information about local 'carrying capacity', dmeographic stochasticity
        effect our inference? 
\end{itemize}




\section{Conclusion}

\clearpage
{
\footnotesize
\bibliography{references}
}



\end{document}
